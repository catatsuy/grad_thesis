\if0
プロセッサなどのハードウェア設計は,アーキテクチャ設計・論理設計・回路設計・物理設計といったフローで行われる.
アーキテクチャ設計と論理設計においては,RTL (Register Transfer Level) のシミュレーションが不可欠である.
このために Verilog HDL などのハードウェア記述言語が用いられることが一般的である.
\fi
\parindent=1.8em
The VLSI chips such as high performance processors or SoCs with processor elements are designed in the flow of architectural design,
logic design, circuit design, and physical design.
In the architectural design and logical design, Register Transfer Level (RTL) simulation is essential for logical verification.
Hardware description languages such as Verilog HDL or VHDL are often used for the RTL modeling and simulation.
\if0
しかし Verilog HDL によるハードウェアのシミュレーションは速度が遅く,ハードウェア設計を行う上で障害になることがあった.
\fi
But the logic simulation speed of Verilog HDL is slow.
% Because of the slow simulation speed, it is difficult to design VLSI chips.

\if0
この問題を解決することを目標に C++ 言語上で RTL モデリングを行う新しい言語である ArchHDL が提案された.
ArchHDL はハードウェアのレジスタを変数,ワイヤを関数として扱うことで,Verilog HDL に近い記述でハードウェアの論理検証を行うことができる.
\fi
ArchHDL which is a new language for hardware RTL modeling on C++ was proposed to solve this problem.
ArchHDL treats a register as a variable and a wire as a function.
The hardware description in ArchHDL is similar to that in Verilog HDL.

\if0
ArchHDL で記述したハードウェアのシミュレーションはオープンソースの Verilog シミュレータである Icarus Verilog と比較してかなり高速である.
最速な有償の Verilog シミュレータである Synopsys 社の VCS と比較しても高速であることが多い.
\fi
The architectural simulation speed with ArchHDL is much faster than
that with Icarus Verilog, which is a verilog simulator of open source.
It is often faster than VCS of Synopsys, Inc.
VCS is one of the fastest verilog simulator and proprietary software.

\if0
本論文では,ArchHDL のさらなる高速化を目指す.
(1)データ変更の有無による条件分岐の除去,(2)値を配列として格納しメモリ配置を工夫,(3)並列化という 3 つの高速化手法を提案し,実装する.
\fi
In this thesis, I aim to speed up the logical simulation with ArchHDL.
I propose and implement three fast methods as follows:
(1) removal of conditional branch for data update,
(2) storing register values to the continuous memory location
and (3) the parallelization of the execution of multiple instances.

\if0
マイクロベンチマークである 4096 個のカウンタ回路を用いた評価では提案手法によりオリジナルの ArchHDL と比較して 5.23 倍高速になった.
現実的なハードウェアであるステンシル計算回路を用いた評価では提案手法によりオリジナルの ArchHDL と比較して 1.95 倍高速になった.
提案手法により今回用いたすべての評価において Synopsys 社の VCS よりも高速にシミュレーションが行えるようになった.
\fi
I elapse the simulation time with 4096 of the counters circuit as a micro benchmark and a stencil-computation circuit as a realistic hardware as an evaluation.
The evaluation results as the micro benchmark show that the elapsed time of the proposed methods is 5.23 times faster than that of the original ArchHDL.
The evaluation results as the realistic hardware show that the elapsed time of the proposed methods is 1.95 times faster than that of the original ArchHDL.
The results show that ArchHDL applied the proposed methods can perform faster than VCS in the all evaluations in this thesis.