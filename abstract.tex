プロセッサなどのハードウェア設計は,アーキテクチャ設計・論理設計・回路設計・物理設計といったフローで行われる.
アーキテクチャ設計と論理設計においては,RTL (Register Transfer Level) のシミュレーションが不可欠である.
このために Verilog HDL などのハードウェア記述言語が用いられることが一般的である.
しかし Verilog HDL によるハードウェアのシミュレーションは速度が遅く,ハードウェア設計を行う上で障害になることがあった.

この問題を解決することを目標に C++ 言語上で RTL モデリングを行う新しい言語である ArchHDL が提案された.
ArchHDL はハードウェアのレジスタを変数,ワイヤを関数として扱うことで,Verilog HDL に近い記述でハードウェアの論理検証を行うことができる.

ArchHDL で記述したハードウェアのシミュレーションはオープンソースの Verilog シミュレータである Icarus Verilog と比較してかなり高速である.
最速な有償の Verilog シミュレータである Synopsys 社の VCS と比較しても高速であることが多い.

本論文では,ArchHDL のさらなる高速化を目指す.
(1)データ変更の有無による条件分岐の除去,(2)値を配列として格納しメモリ配置を工夫,(3)並列化という 3 つの高速化手法を提案し,実装する.

マイクロベンチマークである 4096 個のカウンタ回路を用いた評価では提案手法によりオリジナルの ArchHDL と比較して 5.23 倍高速になった.
現実的なハードウェアであるステンシル計算回路を用いた評価では提案手法によりオリジナルの ArchHDL と比較して 1.95 倍高速になった.
提案手法により今回用いたすべての評価において Synopsys 社の VCS よりも高速にシミュレーションが行えるようになった.
