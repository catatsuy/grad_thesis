ハードウェアの RTL モデリングのための新しい言語として提案している ArchHDL の高速化手法を提案し,実装し,評価した.
ArchHDL ではハードウェアのレジスタを変数,ワイヤを関数として扱うことで,C++ で RTL モデリングを実現する.

高速化手法として (1)データ変更の有無による条件分岐の除去,(2)値を配列として格納しメモリ配置を工夫,
(3)並列化手法を提案した.

提案手法を実装し,ArchHDL を Icarus Verilog と商用ツールである VCS, NC-Verilog の実行時間と比較した.
マイクロベンチマークである 4096 個のカウンタ回路を用いた評価では提案手法によりオリジナルの ArchHDL と比較して 5.23 倍高速になった.
現実的なハードウェアであるステンシル計算回路を用いた評価では提案手法によりオリジナルの ArchHDL と比較して 1.95 倍高速になった.
提案手法により今回用いたすべての評価において Synopsys 社の VCS よりも高速にシミュレーションが行えるようになった.
