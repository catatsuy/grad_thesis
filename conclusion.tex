\if0
ハードウェアの RTL モデリングのための新しい言語として提案している ArchHDL の高速化手法を提案し,実装し,評価した.
ArchHDL ではハードウェアのレジスタを変数,ワイヤを関数として扱うことで,C++ で RTL モデリングを実現する.
\fi
I propose, implement and evaluate the methods for high speed simulation of ArchHDL, which is proposed as a new language for hardware RTL modeling.
ArchHDL treats registers as variables and wires as functions, which realizes an RTL modeling on C++.

\if0
高速化手法として (1)データ変更の有無による条件分岐の除去,(2)値を配列として格納しメモリ配置を工夫,
(3)並列化手法を提案した.
\fi
The three methods I propose and implement in this thesis are as follows:
(1) removal of the conditional branch for data update,
(2) storing register values to the continuous memory location
and (3) the parallelization of the execution of multiple instances.

\if0
提案手法を実装し,ArchHDL を Icarus Verilog と商用ツールである VCS, NC-Verilog の実行時間と比較した.
マイクロベンチマークである 4096 個のカウンタ回路を用いた評価では提案手法によりオリジナルの ArchHDL と比較して 5.23 倍高速になった.
現実的なハードウェアであるステンシル計算回路を用いた評価では提案手法によりオリジナルの ArchHDL と比較して 1.95 倍高速になった.
提案手法により今回用いたすべての評価において Synopsys 社の VCS よりも高速にシミュレーションが行えるようになった.
\fi
In the proposed methods,
I compare the elapsed time of ArchHDL with that of Icarus Verilog, NC-Verilog and VCS
and evaluate the simulation time by use of 4096 of the counters circuit as a micro benchmark and a stencil-computation circuit as a realistic hardware.
With 4096 of the counters circuit, the elapsed time of ArchHDL with the proposed methods is 5.23 times faster than that of the original ArchHDL.
And with a stencil-computation circuit, it is 1.95 times faster.

These results show that ArchHDL applied the three proposed methods can perform faster than VCS in all evaluations in this thesis.
