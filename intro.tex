\if0
\section{研究の背景と目的}
\fi
\section{Background and research objective}

\if0
プロセッサなどのハードウェア設計は,アーキテクチャ設計・論理設計・回路設計・物理設計といったフローで行われる.
アーキテクチャ設計と論理設計においては,RTL (Register Transfer Level) のシミュレーションが不可欠である.
このために Verilog HDL などのハードウェア記述言語が用いられることが一般的である.
\fi
The VLSI chips such as high performance processors or SoCs with processor elements are designed in the flow of architectural design,
logic design, circuit design, and physical design.
In the architectural design and logical design, Register Transfer Level (RTL) simulation is essential for logical verification.
Hardware description languages such as Verilog HDL or VHDL are often used for the RTL modeling and simulation.

\if0
我々は C++ 言語上で RTL モデリングを行う新しい言語である ArchHDL を提案している~\cite{satos:archhdl}.
ArchHDL を用いることで Verilog HDL に近い記述でハードウェアの論理検証を行うことができる.
\fi
\textbf{ArchHDL} was proposed as a new language for hardware RTL modeling~\cite{satos:archhdl}.
The hardware description in ArchHDL is similar to that in Verilog HDL.

\if0
ArchHDL で記述したハードウェアのシミュレーションはオープンソースの Verilog シミュレータである Icarus Verilog~\cite{iverilog}と比較して高速である.
しかし一部のハードウェアシミュレーションにおいて有償の Verilog シミュレータである Cadence 社の NC-Verilog より高速であったが,同じく有償の Synopsys 社の VCS~\cite{vcs} より高速ではないことがあった.
しかし VCS より高速ではないことがあった.
\fi
The architectural simulation speed with ArchHDL is much faster than
that with Icarus Verilog~\cite{iverilog}, which is a verilog simulator of open source.
It is also faster than that with NC-Verilog of Cadence, Inc.
NC-Verilog is a verilog simulator and proprietary software.
However, it is not faster than that with VCS of Synopsys, Inc.~\cite{vcs} in some hardware simulation.
VCS is one of the fastest verilog simulators and proprietary software.

\if0
そこで ArchHDL に分岐の削減,メモリ配置の工夫,OpenMP による並列化といった高速化手法を適用する.
\fi
In this thesis, I aim to speed up the logical simulation with ArchHDL.
I propose and implement three methods as follows:
(1) removal of conditional branch for data update,
(2) storing register values to the continuous memory location
and (3) the parallelization of the execution of multiple instances.

\if0
本論文では Verilog シミュレータの Icarus Verilog, NC-Verilog, VCS と比較して,
今回の高速化手法の有用性を評価する.
Verilog シミュレータである Icarus Verilog, NC-Verilog, VCS と比較して,
今回の高速化手法の有用性を評価する.
\fi
In comparison with Icarus Verilog, NC-Verilog and VCS, I show the usefulness of the methods in this thesis.

\if0
\section{本論文の構成}
\fi
\section{Outline of this thesis}

\if0
本論文の構成は以下の通りである.\ref{c:summary} 章で,ArchHDL の概要を述べる.
\ref{c:method} 章で,ArchHDL の高速化手法を提案する.
\ref{c:evaluation} 章で,様々な Verilog シミュレーションツールと比較して ArchHDL と今回の高速化手法の評価を行う.
\ref{c:conclusion} 章でまとめる.
\fi
The rest of this thesis is organized as follows.
Section \ref{c:summary} provides the overview of ArchHDL.
In section \ref{c:method}, I propose the three methods of ArchHDL.
In section \ref{c:evaluation}, I evaluate the simulation speed of ArchHDL in comparison with the various verilog simulators.
In section \ref{c:conclusion}, I present a conclusion of this thesis.
