\if0
\section{研究の背景と目的}
\fi
\section{Background}

プロセッサなどのハードウェア設計は,アーキテクチャ設計・論理設計・回路設計・物理設計といったフローで行われる.
アーキテクチャ設計と論理設計においては,RTL (Register Transfer Level) のシミュレーションが不可欠である.
このために Verilog HDL などのハードウェア記述言語が用いられることが一般的である.

我々は C++ 言語上で RTL モデリングを行う新しい言語である ArchHDL を提案している~\cite{satos:archhdl}.
ArchHDL を用いることで Verilog HDL に近い記述でハードウェアの論理検証を行うことができる.

ArchHDL で記述したハードウェアのシミュレーションはオープンソースの Verilog シミュレータである Icarus Verilog~\cite{iverilog}と比較して高速である.
しかし一部のハードウェアシミュレーションにおいて有償の Verilog シミュレータである Cadence 社の NC-Verilog より高速であったが,同じく有償の Synopsys 社の VCS~\cite{vcs} より高速ではないことがあった.

そこで ArchHDL に分岐の削減,メモリ配置の工夫,OpenMP による並列化といった高速化手法を適用する.

本論文では Verilog シミュレータの Icarus Verilog, NC-Verilog, VCS と比較して,
今回の高速化手法の有用性を評価する.

\if0
\section{本論文の構成}
\fi
\section{Outline of this thesis}

本論文の構成は以下の通りである.\ref{c:summary} 章で,ArchHDL の概要を述べる.
\ref{c:method} 章で,ArchHDL の高速化手法を提案する.
\ref{c:evaluation} 章で,様々な Verilog シミュレーションツールと比較して ArchHDL と今回の高速化手法の評価を行う.
\ref{c:conclusion} 章でまとめる.