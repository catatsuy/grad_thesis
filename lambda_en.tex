The specification of lambda function is defined in the C++ ISO standard called C++11.
From GCC version 4.5, the lambda function is supported in the C++ standard library.

ArchHDL uses the the lambda function for hardware description.
In this section, we describe about the C++11 lambda function. to the extent necessary to ArchHDL.

\begin{figure}[t]
 \lstinputlisting[language=c++]{src/def_lam.cc.part}
\caption{An example of C++ program which includes a definition of lambda function.}
 \label{src:def_lambda}
\end{figure}

\figref{src:def_lambda} shows an example of C++ program which includes a lambda function.
Note that, the program includes only the definition of a lambda function, so it has no meaning.
The 2nd line of source code is the definition of lambda function which takes two int values as arguments and returns the sum of arguments.

The description of lambda function starts from \textit{lambda-introducer} [] \footnote{
A lambda introducer may contain a \textit{lambda-capture}.
ArchHDL only uses the option [=], so we skip the details of lambda-capture at here.
}
The return type of lambda function is automatically deduced from the return expression.
In the case of \figref{src:def_lambda} line 2, the return type is deduced as int.

\begin{figure}[t]
 \lstinputlisting[language=c++]{src/ex_lam.cc}
 \caption{An example of C++ program which includes a definition of lambda function and a usage of it.}
 \label{src:ex_lambda}
\end{figure}

\figref{src:ex_lambda} shows an example of C++ program using lambda function.
As the 1st line, we need to include the \textit{functional} standard library to use lambda function.

In the 6th and 7th line, the lambda function [=](int x, int y) \{ return x + y; \} is assigned to the function object \textit{Sum}.
The type of the lambda function is std::function$<$int ()$>$, therefore \textit{Sum} is declared so.
Thus, \textit{Sum} becomes a function which takes two int arguments and returns a int value.

In the 8th line, the function \textit{Sum} is called.
The return value of \textit{Sum} is assigned to the variable \textit{c}.
The variable \textit{a} and the variable \textit{b} which they are defined in the 4th and 5th line, are given to the function \textit{Sum} as arguments.

In ArchHDL, the lambda function is used in the way like shown in the \figref{src:ex_lambda}.
