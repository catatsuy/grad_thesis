In ArchHDL, \textit{lambda function} is used for hardware modeling.
%%to understand the description for ArchHDL.
%% You may skip this section if you are familiar with this function.
The specification of lambda function is defined in the C++ ISO standard
named C++11 and
lambda function is supported as a C++ standard library from GCC version 4.5
released in 2010.
In this section,
we explain a little about the C++11 lambda function.

\begin{figure}[t]
 \lstinputlisting[language=c++]{src/def_lam.cc.part}
\caption{A sample C++ program which includes just a definition of lambda function.
The definition of the lambda function is in the line 2.
The function takes two int values of x and y as arguments and returns the sum of these arguments.}
 \label{src:def_lambda1}
\end{figure}

Simply to say, a lambda function is an anonymous function or
a function without its name.
\figref{src:lambda1} shows
a sample program which has a definition of lambda function.
%so it has no meaning.
The definition of the lambda function is in the line 2 of the code.
The function takes two int values of x and y as arguments
and returns the sum of these arguments.
Note that this program includes just a definition of lambda function
for explanation.
Therefore, the function in the line 2 has no effect for this program.
%and nobody uses this function.

The description of lambda function starts from \textit{lambda-introducer} []
and a lambda introducer may contain a \textit{lambda-capture} like ``=''.
%%ArchHDL only uses the description of ``[=]''.
Note that a statement start with ``[=]'' is a lambda function.
%% \footnote{
%% A lambda introducer may contain a \textit{lambda-capture} like ``=''.
%% ArchHDL only uses the option [=], so we skip the details of lambda-capture at here.
%% }.
The return type of a lambda function is automatically deduced from
the return expression in compile time.
In the case of this example, the return type is expected as int.

\begin{figure}[t]
 \lstinputlisting[language=c++]{src/lambda2.cc}
 \caption{
A sample C++ program which includes a definition and a use of lambda function.
Sum is a function object. A lambda function defined in the line 5 is
assigned to Sum. Sum is used in the line 6 with two integer arguments.
This program returns a value of 5.}
 \label{src:lambda2}
\end{figure}

\figref{src:lambda2} shows a sample program using lambda function.
In the line 4 and 5,
a lambda \texttt{function [=](int x, int y) \{ return x + y; \}} is
assigned to a function object \textit{Sum}.
The type of the lambda function is std::function$<$int ()$>$,
and \textit{Sum} is declared with this type.
Thus, \textit{Sum} becomes a function which takes two int arguments
and returns an int value.

In the line 6, the function \textit{Sum} is called.
The return value of \textit{Sum} is assigned to a variable \textit{c}.
The variable \textit{a} and the variable \textit{b}
are defined in the line 2 and 3.
They are two arguments of the function \textit{Sum}.
This program returns a value of 5 computed by \textit{Sum}.
