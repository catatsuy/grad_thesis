\documentclass[12pt,openany,papersize,english]{jsbook}
\usepackage[T1]{fontenc}
\usepackage[utf8]{inputenc}
\usepackage{lmodern}
\usepackage{amssymb,amsmath}
\usepackage{calc}
\usepackage[sc]{mathpazo}
\usepackage[scaled]{helvet}
\usepackage[scaled]{beramono}
\usepackage[bold]{otf}
\usepackage[final]{listings}
\usepackage[bthesis]{archlab}
% \AtBeginDvi{\special{papersize=210truemm,297truemm}}
\graphicspath{{img/}}
\usepackage{textcomp,mediabb,okumacro}
\newcommand{\figref}[1]{Fig.~\ref{#1}}
\newcommand{\tabref}[1]{Table~\ref{#1}}
\usepackage[final]{listings}
\lstset{                 %listingsの設定
numbers=left,            %行番号を左
numberstyle=\scriptsize, %
stepnumber=1,            %1行おきに行番号を
numbersep=1zw,           %ソースと行番号の間隔
lineskip=-0.5zw,         %行間隔 要調整
basicstyle=\ttfamily,     %ttfamily
xleftmargin=15pt,
}
\usepackage{jlisting}
%\newcommand{\unit}[1]{\ifmmode\mathrm{\,[#1]}\else $\mathrm{[#1]}$\fi}%単位に使用 数式中では空白を本文中では空白なし
\title{Research on High-speed Logic Simulation for Computer Architectures}
\authorintitle{Tatsuya~KANEKO}
\nendo{2013}
\studentid{09\_06410}
\advisor{Kenji~KISE}
\affiliation{Department of Computer Science}
\post{Associate Professor}

\kougai{\if0
プロセッサなどのハードウェア設計は,アーキテクチャ設計・論理設計・回路設計・物理設計といったフローで行われる.
アーキテクチャ設計と論理設計においては,RTL (Register Transfer Level) のシミュレーションが不可欠である.
このために Verilog HDL などのハードウェア記述言語が用いられることが一般的である.
\fi
\parindent=1.8em
The VLSI chips such as high performance processors or SoCs with processor elements are designed in the flow of architectural design,
logic design, circuit design, and physical design.
In the architectural design and logical design, Register Transfer Level (RTL) simulation is essential for logical verification.
Hardware description languages such as Verilog HDL or VHDL are often used for the RTL modeling and simulation.
\if0
しかし Verilog HDL によるハードウェアのシミュレーションは速度が遅く,ハードウェア設計を行う上で障害になることがあった.
\fi
But the logic simulation speed of Verilog HDL is slow.
% Because of the slow simulation speed, it is difficult to design VLSI chips.

\if0
この問題を解決することを目標に C++ 言語上で RTL モデリングを行う新しい言語である ArchHDL が提案された.
ArchHDL はハードウェアのレジスタを変数,ワイヤを関数として扱うことで,Verilog HDL に近い記述でハードウェアの論理検証を行うことができる.
\fi
ArchHDL which is a new language for hardware RTL modeling on C++ was proposed to solve this problem.
ArchHDL treats a register as a variable and a wire as a function.
The hardware description in ArchHDL is similar to that in Verilog HDL.

\if0
ArchHDL で記述したハードウェアのシミュレーションはオープンソースの Verilog シミュレータである Icarus Verilog と比較してかなり高速である.
最速な有償の Verilog シミュレータである Synopsys 社の VCS と比較しても高速であることが多い.
\fi
The architectural simulation speed with ArchHDL is much faster than
that with Icarus Verilog, which is a verilog simulator of open source.
It is often faster than VCS of Synopsys, Inc.
VCS is one of the fastest verilog simulators and proprietary software.

\if0
本論文では,ArchHDL のさらなる高速化を目指す.
(1)データ変更の有無による条件分岐の除去,(2)値を配列として格納しメモリ配置を工夫,(3)並列化という 3 つの高速化手法を提案し,実装する.
\fi
In this thesis, I aim to speed up the logical simulation with ArchHDL.
I propose and implement three fast methods as follows:
(1) removal of conditional branch for data update,
(2) storing register values to the continuous memory location
and (3) the parallelization of the execution of multiple instances.

\if0
マイクロベンチマークである 4096 個のカウンタ回路を用いた評価では提案手法によりオリジナルの ArchHDL と比較して 5.23 倍高速になった.
現実的なハードウェアであるステンシル計算回路を用いた評価では提案手法によりオリジナルの ArchHDL と比較して 1.95 倍高速になった.
提案手法により今回用いたすべての評価において Synopsys 社の VCS よりも高速にシミュレーションが行えるようになった.
\fi
I elapse the simulation time with 4096 of the counters circuit as a micro benchmark and a stencil-computation circuit as a realistic hardware as an evaluation.
The evaluation results as the micro benchmark show that the elapsed time of the proposed methods is 5.23 times faster than that of the original ArchHDL.
The evaluation results as the realistic hardware show that the elapsed time of the proposed methods is 1.95 times faster than that of the original ArchHDL.
The results show that ArchHDL applied the proposed methods can perform faster than VCS in the all evaluations in this thesis.}

\setcounter{tocdepth}{3}

\title{Research on High-speed Logic Simulation \\ for Computer Architectures}
\author{Tatsuya KANEKO}
\date{平成 25 年 8 月}

\begin{document}
\maketitle
\parindent=1.8em
\frontmatter

\tableofcontents

\mainmatter

\chapter{Introduction}

\if0
\section{研究の背景と目的}
\fi
\section{Background}

\if0
プロセッサなどのハードウェア設計は,アーキテクチャ設計・論理設計・回路設計・物理設計といったフローで行われる.
アーキテクチャ設計と論理設計においては,RTL (Register Transfer Level) のシミュレーションが不可欠である.
このために Verilog HDL などのハードウェア記述言語が用いられることが一般的である.
\fi
The VLSI chips such as high performance processors or SoC with processor elements are designed in the flow of architectural design,
logic design, circuit design, and physical design.
In the architectural design and logical design, Register Transfer Level (RTL) simulation is essential for logical verification.
Hardware description languages such as Verilog HDL or VHDL are often used for the RTL modeling and simulation.

\if0
我々は C++ 言語上で RTL モデリングを行う新しい言語である ArchHDL を提案している~\cite{satos:archhdl}.
ArchHDL を用いることで Verilog HDL に近い記述でハードウェアの論理検証を行うことができる.
\fi
We proposed \textbf{ArchHDL} as a new language for hardware RTL modeling~\cite{satos:archhdl}.
The hardware description in ArchHDL is similar to that of Verilog HDL.

\if0
ArchHDL で記述したハードウェアのシミュレーションはオープンソースの Verilog シミュレータである Icarus Verilog~\cite{iverilog}と比較して高速である.
しかし一部のハードウェアシミュレーションにおいて有償の Verilog シミュレータである Cadence 社の NC-Verilog より高速であったが,同じく有償の Synopsys 社の VCS~\cite{vcs} より高速ではないことがあった.
しかし VCS より高速ではないことがあった.
\fi
The architectural simulation speed with ArchHDL is much faster than
that with Icarus Verilog~\cite{iverilog} which is a verilog simulator of open source.
It is faster than that with NC-Verilog which is verilog simulators and a paid software.
However, it is not faster than that with VCS which is the fastest in the verilog simulators and a paid software in some hardware simulation.

\if0
そこで ArchHDL に分岐の削減,メモリ配置の工夫,OpenMP による並列化といった高速化手法を適用する.
\fi
In this paper, I aim to further speed up ArchHDL.
(1) Removal of conditional branching due to the presence or absence of data change,
(2) devise a memory allocation and stored as an array value,
(3) parallelization.
I propose and implement fast methods of the three.

\if0
本論文では Verilog シミュレータの Icarus Verilog, NC-Verilog, VCS と比較して,
今回の高速化手法の有用性を評価する.
Verilog シミュレータである Icarus Verilog, NC-Verilog, VCS と比較して,
今回の高速化手法の有用性を評価する.
\fi
In comparison with Icarus Verilog , NC-Verilog, and VCS which are Verilog simulators, I evaluate a usefulness of fast methods in this paper.

\if0
\section{本論文の構成}
\fi
\section{Outline of this thesis}

\if0
本論文の構成は以下の通りである.\ref{c:summary} 章で,ArchHDL の概要を述べる.
\ref{c:method} 章で,ArchHDL の高速化手法を提案する.
\ref{c:evaluation} 章で,様々な Verilog シミュレーションツールと比較して ArchHDL と今回の高速化手法の評価を行う.
\ref{c:conclusion} 章でまとめる.
\fi
The rest of this paper is organized as follows.
In section \ref{c:summary}, we describe about a overview of ArchHDL.
In section \ref{c:method}, we propose fast methods on ArchHDL.
In section \ref{c:evaluation}, we evaluate the simulation speed of ArchHDL in comparison with the various verilog simulators.
Section \ref{c:conclusion} gives the conclusion of this work.


\chapter{Hardware Modeling}

\section{Verilog HDL によるハードウェアモデリング}


\subsection{Verilog HDL の文法}




\subsection{Verilog HDL による記述例}

\section{RTL シミュレーションの高速化の必要性}

\section{関連研究}

MyHDL, JavaRock, SystemC

Verilog HDL などを速くする研究


\chapter{Overview of ArchHDL}

\label{c:summary}

\input{summary}

\chapter{Proposal and Implementation of Fast Methods of ArchHDL}

\label{c:method}

\if0
\section{高速化の方針}
\fi
\section{A Policy of Speeding up}

\if0
高速化の方針として逐次プログラミングにおける最適化と並列化の両方を考える.
\fi
I consider both optimizations in sequential programming and parallelization
as a policy of fast methods.

\if0
\section{逐次プログラムにおける高速化手法}
\fi
\section{Fast methods in sequential program}

\if0
\subsection{データ変更の有無による条件分岐の除去 \label{sss:no_set}}
\fi
\subsection{Removal of conditional branch for data update} \label{sss:no_set}


\if0
\figref{src:reg} に示した実装では,reg クラスのインスタンスの値を更新する方法としてブロッキング代入とノン・ブロッキング代入の 2 つが存在する.
\fi
In the implementation shown \figref{src:reg},
ArchHDL gives non-blocking assignment and blocking assignment as methods of updating the value of the \textit{reg} class instance.

\if0
ブロッキング代入について考える.reg クラスのインスタンスにブロッキング代入が行われた時に curr\_ の値を書き換える.
\fi
I consider blocking assignment.
The member variable \textit{curr\_} of the \textit{reg} class instance must be assigned to the value
if the blocking assignment is carried out to the \textit{reg} class instance.

\if0
一方でノン・ブロッキング代入について考える.
reg クラスのインスタンスにノン・ブロッキング代入が行われた時にメンバ変数 set\_ を true にし,メンバ変数 next\_ に値を代入する.
そして reg::Update 内では set\_ が true の時だけ next\_ をメンバ変数 curr\_ に代入する.
これは reg クラスのインスタンスの値を変更したサイクルのみで,
その reg クラスのインスタンスの値を次サイクルに移る前に新しい値に更新することを意味する.
\fi
I consider non-blocking assignment.
The member variable \textit{set\_} of the \textit{reg} class instance is assigned to true
and the member variable \textit{next\_} is assigned to the value
if the non-blocking assignment is carried out to the \textit{reg} class instance.
Only when the member variable \textit{set\_} is true, the member variable \textit{next\_} is assigned to the value of the member variable \textit{curr\_} in the \textit{Update} function which is a member method of \textit{reg} class.
It shows the value of the register is updated to the new value before the only cycle which is the non-blocking assignment is carried out to the \textit{reg} class instance.

\if0
\figref{src:reg} に示した実装では,
更新されない reg クラスのインスタンスの curr\_ と next\_ の値が同じであるため,代入する処理を行う必要はない.
よって set\_ 変数を用いて不要な代入を避けている.
reg クラスのインスタンスの更新頻度が低い回路であればこの実装が効率的である.
\fi
In the implementation shown \figref{src:reg},
it is not necessary to perform the process of assignment
if the values of the \textit{next\_} and \textit{curr\_} of the \textit{reg} class instance are same.
Therefore the unnecessary assignment is avoided using the variable \textit{set\_}.
This implementation may be effective if the \textit{reg} class instance is rarely updated.

\if0
提案手法について述べる.
この set\_ 変数が true の時のみ代入するのではなく,
次サイクルに移る前に next\_ の値を curr\_ に常に代入するようにする.
こうすることによって分岐のオーバーヘッドが無くなるため,ノン・ブロッキング代入が頻繁に行われる回路で速度向上が期待できる.
\fi
I describe the proposed method.
The value of the \textit{next\_} is always assigned to the variable \textit{curr\_} every cycle.
It eliminates without the overhead of the \textit{if} branch.
This implementation is effective if the \textit{reg} class instance is updated frequently.

% SWoPP より追加

\begin{figure}[tb]
 \lstinputlisting[language=c++]{src/reg_no_set.cc}
\if0
 \caption{条件分岐を除去した reg クラス}
\fi
 \caption{The source code of \textit{reg} class which is removed of the conditional branch}
 \label{src:reg_no_set}
\end{figure}

\if0
\figref{src:reg_no_set} に提案手法の実装を示す.
\figref{src:reg} の実装から set\_ 変数を除いている.
\fi
\figref{src:reg_no_set} shows the implementation of the proposed method.
It is removed of the variable \textit{set\_} from the implementation shown \figref{src:reg}.

% 追加ここまで


\if0
\subsection{値を配列として格納しメモリ配置を工夫} \label{sss:mem_copy}
\fi
\subsection{Storing register values to the continuous memory location} \label{sss:mem_copy}

\begin{figure}[t]
 \centering
 \includegraphics[clip,width=\linewidth-30pt]{registers_orig}
\if0
 \caption{ArchHDL における reg クラスのインスタンスの処理の様子}
\fi
 \caption{The process of the \textit{reg} class instance with ArchHDL}
 \label{fig:regs}
\end{figure}

\if0
\figref{fig:regs} は ArchHDL における reg クラスのインスタンスの処理の様子である.
\figref{src:class_singleton} の 44 行から 46 行の処理を表している.
reg クラスのインスタンスが灰色に塗られており,左からクラスのメタデータ,next\_, curr\_ を表している.
左側の大きな枠が\figref{src:class_singleton} の 18 行の \verb`std::vector` 型の registers\_ である.
実線矢印は代入を表し.点線矢印はポインタ参照を表す.
\fi
\figref{fig:regs} shows the process of the \textit{reg} class instance with ArchHDL.
It denotes \figref{src:class_singleton} in the line from 44 to 46.
The \textit{reg} class instances are painted gray
and metadata of the class, the variable \textit{next\_} and the variable \textit{curr\_} are represented from the left.
The big frame on the left denotes \textit{registers\_} using \verb`std::vector` in the line 18 of \figref{src:class_singleton}.
Solid arrows represent an assignment.
Dotted arrows represent a pointer reference.

\if0
ArchHDL ではノン・ブロッキング代入をシミュレーションするために registers\_ の値を上から順に辿り,
reg クラスの各インスタンスのポインタを取得する.
そして reg クラスの全インスタンスの reg::Update() メソッドを呼ぶ.
\fi
To simulate the non-blocking assignment in ArchHDL,
it follows the values of the \textit{registers\_}
and obtains a pointer to the \textit{reg} class instances.
The \textit{Update} method of the \textit{reg} class instances is called.

\if0
\ref{sss:no_set} 節で述べたデータ変更の有無による条件分岐の除去を行うと reg::Update() メソッド内で行なっている
reg クラスのインスタンスの curr\_ に next\_ の値を代入する処理は
毎サイクル全 reg クラスのインスタンスで実行されることになる.
\fi
If I implement the removal of the conditional branch for data update, which is shown in Section \ref{sss:no_set},
the value of the \textit{next\_} is always assigned to the variable \textit{curr\_} every cycle in \textit{reg::Update} method.

\if0
この代入する処理と reg::Update() メソッドの関数呼び出しの 2 つのオーバーヘッドが ArchHDL の高速化を妨げている.
\fi
The two overheads of this assignment and function call make the speed of the ArchHDL slow down.

\begin{figure}[t]
 \centering
 \includegraphics[clip,width=\linewidth-30pt]{registers_mem}
\if0
 \caption{値を配列として格納しメモリ配置を工夫した reg クラスのインスタンスの処理の様子}
\fi
 \caption{The values of the \textit{reg} class instances stored as an array}
 \label{fig:mem_copy}
\end{figure}

\if0
提案手法について述べる.
\figref{fig:mem_copy} に値を配列として格納しメモリ配置を工夫した reg クラスのインスタンスの処理の様子を示す.
\figref{fig:mem_copy} は提案手法である.
提案手法では全 reg クラスのインスタンスは現在の値と次サイクルの値の実体は持たず,ポインタを保持するように変更している.
\fi
I describe the proposed method.
\figref{fig:mem_copy} shows the proposed method.
\figref{fig:mem_copy} shows the values of the \textit{reg} class instances stored as an array.
It is changed into holding each pointer that all of the \textit{reg} class instances have the value of the next cycle and the value of the current cycle in the proposed method.
\if0
reg クラスのインスタンスが灰色に塗られており,左からクラスのメタデータ,\&next\_, \&curr\_ を表している.
\&next\_, \&curr\_ は next\_, curr\_ のポインタである.
下の枠が next\_, curr\_ の値をまとめた配列であり,
ここでは next collections, curr collections と呼ぶ.
実線矢印は代入を表し,点線矢印はポインタの参照先を表す.
\fi
The \textit{reg} class instances are painted gray
and metadata of the class, the \texttt{\&next\_} and the \texttt{\&curr\_} are represented from the left.
The \texttt{\&next\_} and the \texttt{\&curr\_} are pointers of the value of \textit{next\_} and \textit{curr\_}.
The two big frames on the bottom denote the arrays gathered the value of \textit{next\_} and \textit{curr\_}.
They are named \textit{next collections} and \textit{curr collections} here.
Solid arrows represent an assignment.
Dotted arrows represent a pointer reference.

\if0
ArchHDL の実装では次サイクルに移る前に行われる curr\_ に next\_ の値を代入する処理は
reg クラスのインスタンスが存在するアドレスを調べる必要がある.
しかし値を配列として格納しメモリ配置を工夫すると\figref{fig:mem_copy} に示すように単純な代入となる.
また今まで飛び飛びのアドレスに格納されていた next\_ と curr\_ のメモリ配置がまとまるのでメモリアクセスが連続的に行える.
さらに reg::Update() の関数呼び出しが不要となり,関数呼び出しのオーバーヘッドもなくなる.
これらの理由により高速化が期待できる.
\fi
It is necessary to examine the address to which the \textit{reg} class instances are allocated in the implementation of original ArchHDL
before the process of assigning the value of the \textit{next\_} to the variable \textit{curr\_} is executed before the next cycle.
But if the implementation of the \textit{reg} class is the removal of the conditional branch for data update, \textit{Update} method is a simple assignment as shown in \figref{fig:mem_copy}.
Memory access can be carried out continuously
because the memory allocation of the variable \textit{next\_} and \textit{curr\_} is continuous.
The memory allocation is discrete in the implementation of original ArchHDL.
The overhead of a function call is eliminated
because it is unnecessary to call the \textit{Update} method.
In the above reasoning, this implementation is effective.

\if0
提案手法の実装について述べる.
next collections, curr collections として 2 つの十分大きな \verb/unsigned int/ 型の配列を用意する.
記述された型に応じて,reg クラスのコンストラクタが next\_, curr\_ それぞれの領域を next collections, curr collections に確保する.
確保する領域は参照の高速化のために 4 バイトの倍数とする.
確保された next\_ と curr\_ のアドレスを取得し,インスタンス内の \&next\_, \&curr\_ がそれを保持する.
これまで reg クラスの全インスタンスの reg::Update() メソッドを呼び出していたところを next\_ collections から curr\_ collections の値コピーに変更する.
\fi
I describe the implementation of the proposed method.
Two large arrays of the type of \verb/unsigned int/ are allocated as the \textit{next collections} and the \textit{curr collections}.
The constructor of the \textit{reg} class allocates each region to the \textit{next collections} and the \textit{curr collections} depending on the type of the template arguments.
The allocated region is multiples of 4 bytes to speed up reference.
The \texttt{\&next\_} and the \texttt{\&curr\_} are the addresses of the variable \textit{next\_} and \textit{curr\_}.
The \texttt{reg::Update} method is not called
but the next collections are assigned to the curr collections.


\if0
\subsection{ダブルバッファリング}

これまでの実装では \ref{ss:implementation} 章で述べたように
reg クラスのインスタンスの次サイクルの値が次サイクルに移る前に reg クラスのインスタンスの現在の値に代入される.
そこで偶数回目の実行と奇数回目の実行で次サイクルの値と現在の値を格納している変数をを入れ替えれば
(ダブルバッファリング)代入が減ることが期待できる.

\begin{figure}[t]
 \begin{center}
  \input{img/reg_curr_next}
 \end{center}
 \caption{reg クラスのインスタンスの変数保持の処理の様子}
 \label{fig:reg_curr_next}
\end{figure}

\begin{figure}[t]
 \begin{center}
  \input{img/double_buffer2}
 \end{center}
 \caption{ダブルバッファリングの処理の様子}
 \label{fig:double_buffer}
\end{figure}

\figref{fig:reg_curr_next} はこれまでの ArchHDL の reg クラスのインスタンスの値の保持のイメージである.
読み込み用と書き込み用の変数をそれぞれ保持している.
読み込み用が現在の値であり,書き込み用が次サイクルの値である,
次サイクルに移る前に書き込み用の値が読み込み用の変数に書き込まれる.

\figref{fig:double_buffer}
はダブルバッファリングのイメージである.
奇数回目のサイクルと偶数回目のサイクルで読み込み用と書き込み用の変数を入れ替える.
これにより奇数回目のサイクルで書き込み用であった変数には値が書き込まれているので次サイクルの偶数回目のサイクルで読み込み用として使用出来る.
これを繰り返すことで,次サイクルに移る前に行われる代入処理を無くせる.

しかし今回の手法では reg クラスのインスタンスへ値の書き込みが行われなかった場合に
reg クラスのインスタンスのその時の書き込み用の値に更新が入らない.
次サイクルではその書き込み用の値がそのまま現在の値として使用されるので古い値が使われてしまう.
そのため単純に入れ替えるだけの実装では誤ったシミュレーションを行なってしまう.

また今回の手法はサイクルの回数で依存関係が発生するので \ref{ss:parallel} 節で述べる並列化ができない.

よってライブラリの実装として導入するのは困難であるが,
reg クラスのインスタンスへ常に書き込みが行われるカウンタ回路で試したところ効果があった(具体的な数字).
常に reg クラスのインスタンスに書き込みが行われるハードウェアシミュレーションを逐次処理で行いたい場合には高い効果が期待できる.

以上の理由から本論文ではダブルバッファリングによる評価は行わない.

\fi

\if0
\section{並列化による高速化} \label{ss:parallel}
\fi
\section{The parallelization of the execution of multiple instances} \label{ss:parallel}

\if0
これまで逐次プログラミングにおける高速化を考えてきたが,本節では並列化による高速化について考える.
\fi
I have been considering the fast methods in sequential programming.
However, I consider the speed up by parallelization in this section.

\if0
\figref{src:class_singleton} の 40 行〜 47 行に示すように
毎サイクル,Module クラスと reg クラスの全インスタンスの Module::Always() メソッドと reg::Update() メソッドが呼び出されている.
\fi
As \figref{src:class_singleton} is shown in the line from 40 to 47,
each of the \textit{Module} and \textit{reg} class instances
calls \textit{Module::Always} and \textit{reg::Update} method every cycle.

\if0
\figref{src:class_singleton} の41行〜43行で実行される Module::Always() メソッドはユーザが自由に記述できる.
そのため各 Module クラスのインスタンスで独立に Module::Always() メソッドが実行できる保証はない.
しかしここでは独立に実行できると仮定する.
独立に実行できる条件は今後の研究課題とする.
この場合\figref{src:class_singleton} の41行〜43行に示している Module::Always() メソッドの実行は並列化が可能である.
\fi
\textit{Module::Always} method can be described freely by the user in the line from 41 to 43 in \figref{src:class_singleton}.
Therefore there is no guarantee that \textit{Module::Always} method can be executed separately for each of the \textit{Module} class instances.
However, it is assumed that it can execute independently in this thesis.
It is the research task from now on.
It is possible to parallelize the execution of the \textit{Module::Always} method.

\if0
\figref{src:class_singleton} の 44 行〜 46 行で行われるレジスタの更新は\figref{fig:regs} の実線で表されている.
各インスタンスで独立に行えるので並列化が可能である.
\fi
\figref{fig:regs} shows the update of the register by solid arrows.
It is possible to execute independently for each instance.
It is possible to parallelize the execution of the \textit{reg::Update} method.

\if0
\figref{src:class_singleton} の41行〜47行に示している Module クラスと reg クラスの全インスタンスの Module::Always() メソッドと reg::Update() メソッドの実行は並列化が可能である.
提案手法について述べる.この部分を並列化する.並列化には OpenMP~\cite{openmp} を用いる.
\fi
I describe the proposed method.
I parallelize \textit{Module::Always} and \textit{reg::Update} method in the line from 41 to 46 of \figref{src:class_singleton}.
I use OpenMP~\cite{openmp} to parallelize it.

\begin{figure}[t]
 \lstinputlisting[language=c++]{src/exec_openmp.cc}
\if0
 \caption{Exec メソッド内の for 文を OpenMP で並列化したプログラム}
\fi
 \caption{The source code is parallelized in the \textit{for} statements of \textit{Exec} method with OpenMP in 8 threads.}
 \label{src:exec_openmp}
\end{figure}

\if0
\figref{src:exec_openmp} は\figref{src:class_singleton} の 40 行〜 47 行が 8 スレッドで並列化が行われるように OpenMP 指示文を与えたソースコードである.
2 行目は並列化を何スレッドで行うかを与えている.
今回は 8 をスレッド数に指定している.この数字は環境によって変えることができる.
4 行目と 8 行目は for 文の実行を並列化する OpenMP 指示文である.
\fi
\figref{src:exec_openmp} shows that the source code is parallelized in the \textit{for} statements of \textit{Exec} method with OpenMP in 8 threads.
\figref{src:exec_openmp} shows the line from 40 to 47 of \figref{src:class_singleton} is parallelized.
The number of threads can be changed by the environment.
The line 2 of \figref{src:exec_openmp} gives parallelization in 8 threads.
The line 4 and the line 8 are OpenMP directives to parallelize the execution of the \textit{for} statements.

\if0
一般に,並列化を行う場合は各スレッドに対して均等に負荷を与えることが重要である.
OpenMP では for 文の負荷を分散するスケジュール方法として,静的に決定する static や,動的に決定する dynamic など複数の方法が存在する.
一般に,スケジュール方法に dynamic を指定した場合,各スレッドに割り当てられる負荷が均等に近くなるメリットはあるが,オーバーヘッドが大きいデメリットがある.
ArchHDL により記述されたハードウェア記述では各モジュールと各レジスタの負荷が大幅に変わらないのであれば,static を指定した方が効率が良いと考えられる.
\fi
Generally, it is important for parallelization to assign the tasks equally to each thread.
Several methods such as \textit{static} to determine statically
and \textit{dynamic} to determine dynamic exist
as a way to balance the burden of the \textit{for} statement in OpenMP.
If you specify the \textit{dynamic} as a way to schedule,
it has the advantage that the tasks is assigned to each thread almost equally.
But It has the disadvantage of large overhead.
If each of the tasks of each register and each module is not very different in the hardware description with ArchHDL,
it is almost effective to specify \textit{static}.

\if0
他にも OpenMP にはオプションとしてチャンクサイズを指定できる.
スケジュール方法を static にし,チャンクサイズを指定しなければ,チャンクサイズはループの反復数をスレッド数で割った商とほぼ同じ値になる.
\fi
The chunk size can be specified as the option of the OpenMP.
If you specify the \textit{static} as a way to schedule and no chunk size,
the chunk size is similar to a value dividing the number of loop iterations by the number of threads.

\if0
今回の評価ではスケジュール方法は static でチャンクサイズは指定しないデフォルトの設定で行う.
私はスケジュール方法として static を,チャンクサイズとしてなしを指定する.
これはデフォルトの設定である.
\fi
I specify the \textit{static} as a way to schedule and no chunk size in the evaluation of this paper.
It is the default setting.


\chapter{Evaluation}

\label{c:evaluation}

\if0
本章では ArchHDL での論理シミュレーションの実行時間を評価し,Icarus Verilog, NC-Verilog, VCS での論理シミュレーションの実行時間と比較する.
\fi
In this chapter, I evaluate the elapsed time of logic simulation with ArchHDL and it is compared with that of logic simulation with Icarus Verilog, NC-Verilog and VCS.

\begin{table}[t]
\if0
 \caption{実行環境}
\fi
 \caption{The simulation environment}
 \label{table:exec_env}
 \begin{center}
  \begin{tabular}{l|c|c} \hline
         &  Icarus Verilog, ArchHDL  &  NC-Verilog, VCS   \\ \hline
  OS     &  Ubuntu12.04             &  CentOS5.9        \\
  CPU    &  Core i7-3770K 3.50GHz   &  Core i7-3770K 3.50GHz  \\
\if0
  メモリ  &  $16\,\mathrm{GB}$       &  $16\,\mathrm{GB}$  \\ \hline
\fi
   Memory &  $16\,\mathrm{GB}$       &  $16\,\mathrm{GB}$  \\ \hline
  \end{tabular}
 \end{center}
\end{table}

\if0
\tabref{table:exec_env} に実行環境をまとめる.
評価には同じ仕様の 2 台の計算機を用いる.
一台は Icarus Verilog, ArchHDL の評価に用いる.もう一台は NC-Verilog, VCS の評価に用いる.
CPU,メモリなどのハードウェアの仕様は同一であるがソフトウェアの制約により異なる OS を利用する.
\fi
\tabref{table:exec_env} shows the simulation environment.
I use the two computers of the same specification for the evaluation.
One computer is used to evaluate the simulation time with Icarus Verilog and ArchHDL.
The other is used to evaluate that with NC-Verilog and VCS.
The two computers are the same specification such as hardware, CPU, memory and so on.
However, I use different operating systems due to the limitations of the software.

\if0
異なる OS を用いる理由を述べる.
NC-Verilog と VCS は RedHat 系のディストリビューションのみをサポートしている.
今回は RedHat 系のディストリビューションである CentOS5.9 を用いる.
しかし CentOS5.9 に含まれる gcc のバージョンは 4.1.2 である.
\ref{s:lambda}節で述べたように,ArchHDL では C++11 のラムダ関数を用いて記述するため gcc のバージョンは 4.5 以上が必要である.
その条件を満たす Ubuntu12.04 を評価に用いる.
Ubuntu12.04 に含まれる gcc のバージョンは 4.6.3 である.
gcc の最適化オプションとして \verb/-O2/ を用いる.
Icarus Verilog はどちらのディストリビューションでも動作するが,今回は Ubuntu12.04 を用いる.
Ubuntu12.04 に含まれる Icarus Verilog のバージョンは 0.9.5 である.
VCS のバージョンは vcsC-2009.06 を用いる.NC-Verilog のバージョンは 06.20-s004 を用いる.
\fi
I describe the reason for using different operating systems.
NC-Verilog and VCS support only RPM-based Linux distributions.
I use CentOS5.9 which is RPM-based Linux distributions for this evaluation.
However, the version of GCC includes CentOS5.9 is 4.1.2.
Section \ref{s:lambda} shows that the version of GCC requires 4.5 or greater for using the lambda function in ArchHDL.
I also use to evaluate the Ubuntu12.04 because the version of GCC includes Ubuntu12.04 is 4.6.3.
I use \verb/-O2/ as optimization option of GCC.
Icarus Verilog can run on both computers.
I use it on Ubuntu12.04 in this evaluation.
The version of Icarus Verilog includes Ubuntu12.04 is 0.9.5.
The version of NC-Verilog is 06.20-s004.
The version of VCS is vcsC-2009.06.

\if0
今回用いる計算機の CPU の物理コアは 4 コアであるため, OpenMP による並列化はスレッド数を 8 個にして評価する.
\fi
Because physical cores of the CPU of the computers are 4 cores for this evaluation,
I evaluate the parallelization with OpenMP in 8 threads.

\if0
評価では,2 つのマイクロベンチマークと,現実的なハードウェアのベンチマークとしてステンシル計算回路\cite{koba:stencil}を用いる.
Verilog HDL と ArchHDL のためのハードウェアシミュレーションは手作業により作成した.
両ハードウェアシミュレーションの出力は同様になるように作成した.
\fi
In the evaluation, I use two micro benchmarks
and stencil-computation circuit as a benchmark of hardware realistic.
I have created hardware simulation for ArchHDL and Verilog HDL by myself
and the outputs of both the hardware simulation are same.

\if0
評価結果に用いているラベルの名前について述べる.
オリジナルの ArchHDL は \textbf{ArchHDL} と表す.
\ref{sss:no_set} 節で述べた条件分岐の除去を適用したものを \textbf{NO SET} と表す.
\ref{sss:mem_copy} 節で述べたメモリ配置の工夫を適用したものを \textbf{MEM MAP} と表す.
\ref{ss:parallel} 節で述べた並列化を行ったものを \textbf{PARA} と表す.
メモリ配置の工夫と並列を同時に適用したものを \textbf{MEM MAP + PARA} と表す.
\fi
I describe the name of the labels which are used on the evaluation results.
Original ArchHDL is named \textbf{ArchHDL}.
It is named \textbf{NO SET} that I apply removal of the conditional branch for data update to ArchHDL, which is described in Section \ref{sss:no_set}.
It is named \textbf{MEM MAP} that I apply storing register values to the continuous memory location to ArchHDL, which is described in Section \ref{sss:mem_copy}.
It is named \textbf{PARA} that I apply the parallelization of the execution of multiple instances to ArchHDL, which is described in Section \ref{ss:parallel}.


\section{Evaluation by Micro Benchmark}

\if0
マイクロベンチマークとしてカウンタ回路と XORSHIFT による乱数生成回路を用いる.
\fi
I evaluate the simulation time with the counter circuits and the random number generator circuits by XORSHIFT RNG as micro benchmarks.

\begin{figure}[tb]
 \lstinputlisting[language=c++]{src/xorshift_alg.cc}
\if0
 \caption{XORSHIFT 法に基づく乱数生成のアルゴリズム}
\fi
 \caption{The algorithm of random number generation based on XORSHIFT RNG on C language}
 \label{src:xorshift_alg}
\end{figure}

\if0
カウンタ回路とは \figref{src:counter} に示した 1 サイクルごとに 1 を足す回路である.
ハードウェアの規模を増やすためにカウンタの数を指定できるようにした.
XORSHIFT による乱数生成回路とはシフトと XOR 演算のみで構成できる XORSHIFT 法に基づく乱数生成器をハードウェア記述によって実装した回路である.
\figref{src:xorshift_alg} に XORSHIFT 法に基づく乱数生成のアルゴリズムを C 言語によって実装したものを示す.
\fi
The counter circuit is a circuit that adds 1 per cycle as shown in \figref{src:counter}.
I can specify the number of the counter in order to increase the scale of hardware.
The random number generator circuits by XORSHIFT RNG is a circuit which is implemented as the random number generator based on XORSHIFT RNG by a hardware description.
XORSHIFT RNG use only the \textit{exclusive or} and a \textit{bit shifted} to generate the random numbers.
\figref{src:xorshift_alg} shows the algorithm of random number generator based on XORSHIFT RNG on C language.

\begin{figure}[tb]
 \centering
 \includegraphics[clip,width=\linewidth]{counter_4096}
\if0
 \caption{4096 個のカウンタ回路の実行時間を Icarus Verilog と比較した速度向上比}
\fi
 \caption{Speed up ratio compared with the elapsed time of 4096 of the counter circuit with Icarus Verilog}
 \label{fig:counter4096}
\end{figure}

\if0
\figref{fig:counter4096} に 4096 個のカウンタ回路の実行時間を Icarus Verilog と比較した速度向上比を示す.
縦軸は Icarus Verilog での実行時間を 1 とした速度向上比を示している.
\fi
\figref{fig:counter4096} shows the speed up ratio compared with the elapsed time of 4096 of the counter circuit with Icarus Verilog.
The speed up ratio as elapsed time in Icarus Verilog is 1 is denoted in the vertical axis.

\if0
ArchHDL は商用の NC-Verilog, VCS と比較してもかなり高速である.
\textbf{MEM MAP + PARA} の論理シミュレーション実行時間は NC-Verilog の 58.8 倍,VCS の 56.7 倍高速である.
\fi
The simulation time of ArchHDL is much faster than that of NC-Verilog and VCS.
The evaluation result shows that the elapsed time of \textbf{MEM MAP + PARA} is 58.8 times faster than that of NC-Verilog and 56.7 times faster than that of VCS.

\if0
また今回提案している高速化手法はオリジナルの ArchHDL に比べていずれも効果が出ている.
\textbf{MEM MAP + PARA} の論理シミュレーション実行時間はオリジナルの ArchHDL の 5.23 倍高速である.
\fi
The proposed methods in this paper are effective as compared with the original ArchHDL.
The elapsed time of \textbf{MEM MAP + PARA} is 5.23 times faster than that of the original ArchHDL.



\begin{figure}[tb]
 \centering
 \includegraphics[clip,width=\linewidth]{counter_con}
\if0
 \caption{高速化手法を適用した ArchHDL と OpenMP を適用したカウンタ回路の実行時間を Icarus Verilog と比較した速度向上比}
\fi
 \caption{Speed up ratio compared with the elapsed time of the counter circuits with Icarus Verilog in ArchHDL applying the method and OpenMP}
 \label{fig:counter_con}
\end{figure}

\if0
\figref{fig:counter_con} に高速化手法を適用した ArchHDL と OpenMP を適用したカウンタ回路の実行時間を Icarus Verilog と比較した速度向上比を示す.
縦軸は Icarus Verilog での実行時間を 1 とした速度向上比を示している.
横軸はカウンタの個数である.
\fi
\figref{fig:counter_con} shows the speed up ratio compared with the elapsed time of the counter circuits with Icarus Verilog in ArchHDL applying the method and OpenMP.
The speed up ratio as elapsed time in Icarus Verilog is 1 is denoted in the vertical axis.
The number of counters is denoted in horizontal axis.

\if0
\textbf{MEM MAP} は逐次に実行されているので Icarus Verilog と比較した速度向上比はカウンタの個数を変えてもほとんど変わらない.
並列化を行った \textbf{PARA} と \textbf{MEM MAP + PARA} はカウンタの個数が 1024 個以上で \textbf{MEM MAP} よりも高速になる.
\textbf{PARA} より \textbf{MEM MAP + PARA} の方が常に高速であるので今回提案している逐次処理での高速化手法は並列化を行った場合でも効果が出ている.
カウンタの個数はハードウェアの規模とみなせるため,並列化が有効なのはある程度規模の大きい回路であると言える.
\fi
The speed up ratio compared to Icarus Verilog is almost unchanged if you change the number of the counters because \textbf{MEM MAP} is executed in sequential program.
The elapsed time of \textbf{PARA} and \textbf{MEM MAP + PARA} which are executed in parallel is faster than that of \textbf{MEM MAP} if the number of counters is 1024 or more.
The proposed methods in sequential program are effective in parallel
because the simulation time of \textbf{MEM MAP + PARA} is always faster than that of \textbf{PARA}.
Because the number of counters can be regarded as the scale of hardware, the parallelization is effective if the hardware is large-scale.


\begin{figure}[tb]
 \centering
 \includegraphics[clip,width=\linewidth]{xorshift}
\if0
 \caption{512 個の XORSHIFT による乱数生成器の実行時間を Icarus Verilog と比較した速度向上比}
\fi
 \caption{Speed up ratio compared with the elapsed time of 512 of the random number generator circuits by XORSHIFT RNG with Icarus Verilog}
 \label{fig:xorshift}
\end{figure}

\if0
\figref{fig:xorshift} は XORSHIFT による乱数生成器での実行時間を Icarus Verilog と比較した速度向上比である.
試行回数は 524,288 回である.初期値の異なる乱数生成器を 512 個用意している.
\fi
\figref{fig:xorshift} shows speed up ratio compared with the elapsed time of 512 of the random number generator circuits by XORSHIFT RNG with Icarus Verilog.
The number of trials is 524,288 times.
I use the 512 random number generators with different initial values.

\if0
ArchHDL は商用の NC-Verilog, VCS と比較してもかなり高速である.
\textbf{MEM MAP + PARA} の論理シミュレーション実行時間は NC-Verilog の 32.2 倍,VCS の 11.3 倍高速である.
\fi
The simulation time of ArchHDL is much faster than that of NC-Verilog and VCS.
The simulation time of \textbf{MEM MAP + PARA} is 32.2 times faster than that of NC-Verilog and 11.3 times faster than that of VCS.

\if0
また今回提案している高速化手法はオリジナルの ArchHDL に比べていずれも効果が出ている.
\textbf{MEM MAP + PARA} の論理シミュレーション実行時間はオリジナルの ArchHDL の 2.78 倍高速である.
\fi
The proposed methods in this paper are effective as compared with the original ArchHDL.
The simulation time of \textbf{MEM MAP + PARA} is 2.78 times faster than that of the original ArchHDL.


\section{Evaluation by Stencil-Computation Circuit}

\if0

\begin{table}[tb]
 \caption{ステンシル計算回路でのプロファイリング結果 1.1}
 \label{table:stencil_prof1.1}
 \begin{center}
  % \setlength{\tabcolsep}{3pt}
  \begin{tabular}{lr} \toprule
  関数名 & 実行時間に占める割合 (\%) \\ \midrule
  reg::Update() (合計) & 16.57 \\
  ArchHDL::Step() & 12.47 \\
  brk & 15.05 \\ \bottomrule
  \end{tabular}
 \end{center}
\end{table}

\fi

\begin{figure}[tb]
 \centering
 \includegraphics[clip,width=\linewidth]{stencil}
\if0
 \caption{ステンシル計算回路の Icarus Verilog と比較した実行時間の速度向上比}
\fi
 \caption{Speed up ratio compared with the elapsed time of a stencil-computation circuit with Icarus Verilog}
 \label{fig:stencil}
\end{figure}

\if0
\figref{fig:stencil} はステンシル計算回路での実行結果である.
縦軸は Icarus Verilog と比較したそれぞれの速度向上比である.
\fi
\figref{fig:stencil} shows speed up ratio compared with the elapsed time of a stencil-computation circuit with Icarus Verilog.
The speed up ratio as elapsed time in Icarus Verilog is 1 is denoted in the vertical axis.

\if0
オリジナルの ArchHDL は商用の NC-Verilog より高速であったが,同じく商用の VCS はオリジナルの ArchHDL と \textbf{NO SET} より高速である.
しかし逐次実行での高速化手法と並列化を共に適用した \textbf{MEM MAP + PARA} の論理シミュレーション実行時間は VCS の 1.83 倍高速である.
\fi
The simulation time of the original ArchHDL is faster than that of NC-Verilog.
The simulation time of the original ArchHDL and \textbf{NO SET} is not faster than that of VCS.
But the simulation time of \textbf{MEM MAP + PARA} is 1.83 times faster than that of VCS.

\if0
ステンシル計算回路の場合は Update() は 325,469,175 回呼ばれているのに対して,
reg の値に更新がないのは 5,145,760 回である.
つまり更新がないのは Update() メソッド呼び出し全体の $1.58\%$ 程度に過ぎない.
それにより条件分岐を無くす \textbf{NO SET} の論理シミュレーションはオリジナルの ArchHDL より高速である.
また Update() のメソッド呼び出しを減らし,かつメモリ配置を工夫している \textbf{MEM MAP} の論理シミュレーション実行時間はオリジナルの ArchHDL の 1.31 倍高速である.
\fi
\textit{Update} method is called 325,469,175 times in a stencil-computation circuit.
The number of the value of the reg updated is 5,145,760 times.
That is, the number of the value of the reg not updated is not only $1.58\%$ of the entire \textit{Update} method call.
Therefore the simulation time of \textbf{NO SET} is faster than that of the original ArchHDL.
The simulation time of \textbf{MEM MAP} is 1.31 times faster than that of the original ArchHDL.

\if0
また Module が 133 個,reg が 991 個存在する回路なので並列化の効果も大きい.
逐次実行での高速化手法と並列化を共に適用した \textbf{MEM MAP + PARA} の論理シミュレーション実行時間はオリジナルの ArchHDL の 1.95 倍高速である.
\fi
The parallelization is effective because this hardware with ArchHDL has the 133 \textit{Module} class instance and the 991 \textit{reg} class instance.
The simulation time of \textbf{MEM MAP + PARA} is 1.95 times faster than that of the original ArchHDL.


% \subsection{高速化の解析}


\chapter{Conclusion}

\label{c:conclusion}

\if0
ハードウェアの RTL モデリングのための新しい言語として提案している ArchHDL の高速化手法を提案し,実装し,評価した.
ArchHDL ではハードウェアのレジスタを変数,ワイヤを関数として扱うことで,C++ で RTL モデリングを実現する.
\fi
I propose, implement and evaluate the fast methods of ArchHDL which we propose as a new language for hardware RTL modeling.
ArchHDL treats registers as variables and wires as functions.
It realizes an RTL modeling on C++.

\if0
高速化手法として (1)データ変更の有無による条件分岐の除去,(2)値を配列として格納しメモリ配置を工夫,
(3)並列化手法を提案した.
\fi
I propose and implement fast methods of the three.
(1) Removal of conditional branching due to the presence or absence of data change,
(2) devising a memory allocation and stored as an array value,
(3) parallelization.

\if0
提案手法を実装し,ArchHDL を Icarus Verilog と商用ツールである VCS, NC-Verilog の実行時間と比較した.
マイクロベンチマークである 4096 個のカウンタ回路を用いた評価では提案手法によりオリジナルの ArchHDL と比較して 5.23 倍高速になった.
現実的なハードウェアであるステンシル計算回路を用いた評価では提案手法によりオリジナルの ArchHDL と比較して 1.95 倍高速になった.
提案手法により今回用いたすべての評価において Synopsys 社の VCS よりも高速にシミュレーションが行えるようになった.
\fi
I implement the proposed methods. I compare the elapsed time of ArchHDL with that of Icarus Verilog, NC-Verilog and VCS.
When I evaluated using 4096 of the counter circuit as a micro benchmark,
evaluation time of the proposed methods is 5.23 times faster than that of the original ArchHDL.
When I evaluated using a stencil-computation circuit as a realistic hardware,
evaluation time of the proposed methods is 1.95 times faster than that of the original ArchHDL.
The proposed methods can be performed faster than VCS in the all evaluations here.

\backmatter

\chapter{Acknowledgment}

\label{c:acknowledgment}

\if0
研究を進めるにあたり.適切な指導をしていただいた指導教員の吉瀬謙二准教授に感謝します.吉瀬研究室の皆様にも数々の助言をいただき,大変お世話になりました.特に吉瀬研究室の博士課程の佐藤真平さんには多大な貢献をしていただきました.また同じく博士課程の高前田(山崎)伸也さんと笹河良介さんにも数多くの助言をしていただきました.
入谷優さんにも論文の構成など数多くの助言をしていただきました.
ArchHDL の開発に多大な貢献をしていただいた佐野伸太郎さんに感謝します.
\fi
I would like to express the deepest appreciation to Associate Prof.~Kenji KISE.\@
He has been my supervisor. His constant support, guidance, and encouragement have been essential for me to complete my thesis.
I also would like to thank all the members at Kise Laboratory.
I would particularly like to thank Mr.~Shimpei SATO.\@
He proposed ArchHDL and gived insightful comments and suggestions.
Discussions with Mr.~Shinya TAKAMAEDA-YAMAZAKI and Mr.~Ryosuke SASAKAWA have been insightful. I would like to thank them.

I would like to thank Ms.~Yukiko ASOH.\@
She corrected the English written by me.

I would like to thank Mr.~Masaru IRITANI.\@
I received generous support from him.

I would like to thank Mr.~Shintaro SANO.\@
He had made a significant contribution to the development of ArchHDL.


\label{c:relatedwork}

\input{relatedwork}

\end{document}
